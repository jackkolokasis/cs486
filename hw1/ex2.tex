\section*{Exercise B}
\paragraph{1.}
If we delete the the line \textbf{pred.next = curr} then we will have
the following possible scenarios: (1) the thread is going through
nodes that have been removed and thus, pred can point to a node
already removed. (2) It possible for new nodes to be added before the
current \textbf{pred.next} thus, the validation would not have the
correct status.

\paragraph{2}
In the case of the Lazy algorithm, canceling the flag marked
invalidates the validate() and contains() method that depend on this
mark.

\paragraph{3}
No, we cannot modify the insert() method of the fine-grained
synchronization list so that it locks only on node because we need to
lock the previous node and the current node each time. During the
insertion path the previous node will point to the new node and the
new node will point to the current node. In the possibility of keeping
only one lock in the current node then two threads might try to add a
new node to the previous node which that is not correct. 
