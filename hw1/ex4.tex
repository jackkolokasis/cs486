\section*{Exercise 7C}

\begin{lstlisting}[style=mycode]

// Node structure
struct Node {
	int _key;               // Node key
	boolean marked;			// Marked bit
	struct Node *left;      // Left child
	struct Node *right;     // Right child
	struct Node *l_thread;  // Left threaded
	struct Node *r_thread;  // Right threaded
};

boolean validate(Node *par, Node *curr)
{
	return !par->marked && !curr->marked;
}

// Insert a Node in Binary Threaded Tree 
struct Node *BSTInsert(struct Node *root, int key) 
{ 
    Node *ptr; 
    Node *par = NULL;		// Parent node
	Node *child = NULL;	    // Child node

    // Searching for a Node with given value 
	lock(root);
	ptr = root

    while (ptr != NULL) 
    { 
        // If key already exists, return 
        if (key == (ptr->key)) 
        { 
			unlock(root);
            return ptr; 
        } 
  
        par = ptr;	// Update parent pointer 
  
        // Moving on left subtree. 
        if (key < ptr->key) 
        { 
            if (ptr->l_thread == false) 
				child = ptr->left
                ptr = child
            else
                break; 
        } 
  
        // Moving on right subtree. 
        else
        { 
            if (ptr->r_thread == false) 
				child = ptr->right;
                ptr = child;
            else
                break; 
        } 
    } 

	lock(par)
	lock(ptr)
  
    // Create a new node 
    Node *tmp = new Node; 
    tmp->key = key; 
    tmp->l_thread = true; 
    tmp->r_thread = true; 
  
    if (par == NULL) 
    { 
        root = tmp; 
        tmp->left = NULL; 
        tmp->right = NULL; 
		unlock (par);
		return root;
    } 
    else if (key < (par->key)) 
    { 
		if (validate(par, ptr))
		{
			tmp->left = par->left; 
			tmp->right = par; 
			par->l_thread = false; 
			par->left = tmp; 
		}
    } 
    else
    { 
		if (validate(par, ptr))
		{
			tmp->left = par; 
			tmp->right = par->right; 
			par->r_thread = false; 
			par->right = tmp; 
		}
    } 

	unlock(par)
	unlock(ptr)
  
    return root; 
} 

// Insert a Node in Binary Threaded Tree 
struct Node *BSTSearch(struct Node *root, int key) 
{ 
    Node *ptr;
    Node *par;
    Node *child; 

    // Searching for a Node with given value 
	ptr = root

    while (ptr != NULL) 
    { 
        // If key already exists, return 
        if (key == (ptr->key)) 
        { 
			return ptr && !ptr. 
        } 

		if (!ptr->marked)
			return ptr;
		else 
			return NULL;
  
        // Moving on left subtree. 
        if (key < ptr->key) 
        { 
            if (ptr->l_thread == false) 
				child = ptr->left
                ptr = child
            else
                break; 
        } 
  
        // Moving on right subtree. 
        else
        { 
            if (ptr->r_thread == false) 
				child = ptr->right;
                ptr = child;
            else
                break; 
        } 
    } 
} 

// Delete a Node in Binary Threaded Tree 
struct Node *BSTDelete(struct Node *root, int key) 
{ 
    Node *ptr; // Current pointer
    Node *par; // Parent pointer
    Node *child; // Child pointer
	int found = 0;	// Key is found 

    // Searching for a Node with given value 
	lock(root);
	ptr = root

	// Search the key
    while (ptr != NULL) 
    { 
		if (dkey == ptr->key)
		{
			found = 1
			break;
		}

		par = ptr

		if (dkey < ptr->key)
		{
			if (ptr->l_thread == false)
			{
				child = ptr->left;
				ptr = child;
			}
			else
				break;
		}
		else {
			if (ptr->r_thread == false)
			{
				child = ptr->right;
				ptr = child;
			}
			else
				break;
		}
	} 
	
	lock(ptr)
	lock(par)

	if (found == 0)
		printf("The key does not found");

	else if (ptr->l_thread == false && ptr->r_thread == false)
	{
		struct Node* par_succ = ptr;
		struct Node* succ = ptr->right;

		while (succ->left != NULL)
		{
			parsucc = succ;
			succ = succ->left;
		}

		lock(suck)

		if (validate(ptr, suck))
		{
			ptr->key = succ->key
			succ->marked = true
		}
	}

	else if (ptr->l_thread == false)
	{
		if (ptr->l_thread == false)
		{
			child = ptr->left
			lock(child)
		}
		else {
			child = ptr->right
			lock(child)
		}

		if (par == NULL)
			root = child;
		else if (ptr == par->left)
		{
			par->left = child;
		}
		else {
			par->right = child
		}

		Node *s = find_successor(ptr)
		lock(s)
		Node *p = find_predecessor(ptr)
		lock(p)

		if (validate(p, s))
		{
			// if ptr has left subtree
			if (ptr->l_thread == false)
				p->right = s

			// if ptr has right subtree
			else {
				if (ptr->r_thread == false)
					s->left = p
			}

			ptr->marked = true;
		}

		free(ptr)
		unlock(par)
		unlock(child)
		unlock(p)
		unlock(s)

		return root;
	}

	else if (ptr->r_thread == false)
	{
		if (ptr->l_thread == false)
		{
			child = ptr->left
			lock(child)
		}
		else {
			child = ptr->right
			lock(child)
		}

		if (par == NULL)
			root = child;
		else if (ptr == par->left)
		{
			par->left = child;
		}
		else {
			par->right = child
		}

		Node *s = find_successor(ptr)
		lock(s)
		Node *p = find_predecessor(ptr)
		lock(p)
		
		if (validate(p, s))
		{
			// if ptr has left subtree
			if (ptr->l_thread == false)
				p->right = s

			// if ptr has right subtree
			else {
				if (ptr->r_thread == false)
					s->left = p
			}

			ptr->marked = true;
		}

		free(ptr)
		unlock(par)
		unlock(child)
		unlock(p)
		unlock(s)

		return root;

	}
	else
	{
		// Delete root 
		struct Node* par_successor = ptr;
		struct Node* successor = ptr->right;
		lock(successor)

		while(successor->left != NULL)
		{
			par_successor = successor
			successor = successor->left

		}

		lock(par_successor)
		lock (successor)

		ptr->key = successor->key

		if (successor->l_thread == true && successor->r_thread == true)
		{
			struct Node* child;

			if (successor->l_thread == false)
			{
				child = successor->left
				lock(child)
			}
			else {
				child = successor->right
				lock(child)
			}

			if (par_successor == NULL)
				root = child;
			else if (successor == successor->left)
			{
				par_successor->left = child;
			}
			else {
				par_successor->right = child
			}

			Node *s = find_successor(successor)
			lock(s)
			Node *p = find_predecessor(successor)
			lock(p)

			if(validate(p, s))
			{
				// if successor has left subtree
				if (successor->l_thread == false)
					p->right = s

				// if successor has right subtree
				else {
					if (successor->r_thread == false)
						s->left = p
				}

				ptr->marked = true;
			}

			free(ptr)
			unlock(par)
			unlock(successor)
			unlock(par_successor)
			unlock(child)
			unlock(p)
			unlock(s)

			return root;
		}
		// If the node to be deleted is left or its parent node
		else 
		{
			stuct Node* child
			if (successor->l_thread == false)
			{
				child = successor->left
				lock(child)
			}
			else {
				child = successor->right
				lock(child)
			}

			if (par_successor == NULL)
				root = child;
			else if (successor == par_successor->left)
			{
				par_successor->left = child;
			}
			else {
				par_successor->right = child
			}

			Node *s = find_successor(successor)
			lock(s)
			Node *p = find_predecessor(successor)
			lock(p)

			if(validate(p, s))
			{
				// if ptr has left subtree
				if (successor->l_thread == false)
					p->right = s

				// if ptr has right subtree
				else {
					if (successor->r_thread == false)
						s->left = p
				}
				ptr->marked = true;
			}

			free(ptr)
			unlock(par)
			unlock(child)
			unlock(p)
			unlock(s)
			unlock(successor)
			unlock(par_successor)

			return root;
				
			}
	}

\end{lstlisting}

\paragraph{2.}
We implement a lazy synchronization locking for the lock-based
implementation of the concurrent threaded binary search tree.  In add,
search, remove operation we scan through the tree without acquiring
any locks, traversing both marked and umarked nodes.  When we are
going to insert or delete a node then we lock the parent, current, and
child pointers and we check if they are marked. In the case that these
nodes are marked then these nodes are not valid. So, we do not perform
the operation.

\paragraph{3.}
After unlocking operations we could have liniarization points.

\paragraph{4.}
The algorithm is correct because we ensure using validate() method
that a node that we make operations is not marked as deleted node.

\paragraph{5.}
The algorithm is deadlock-free and starvation-free because threads
could traverse the list and do not wait on locked nodes. Also this
implementation allows threads to work on different nodes of the tree.

\paragraph{6.}
