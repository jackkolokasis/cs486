\section*{Exercise 7C}

\begin{lstlisting}[style=mycode]

// Node structure
struct Node {
	int _key;               // Node key
	boolean marked;			// Marked bit
	struct Node *left;      // Left child
	struct Node *right;     // Right child
	struct Node *l_thread;  // Left threaded
	struct Node *r_thread;  // Right threaded
};

boolean validate(Node *par, Node *curr)
{
	return !par->marked && !curr->marked;
}

// Insert a Node in Binary Threaded Tree 
struct Node *BSTInsert(struct Node *root, int key) 
{ 
    Node *ptr; 
    Node *par = NULL;		// Parent node
	Node *child = NULL;	    // Child node

    // Searching for a Node with given value 
	lock(root);
	ptr = root

    while (ptr != NULL) 
    { 
        // If key already exists, return 
        if (key == (ptr->key)) 
        { 
			unlock(root);
            return ptr; 
        } 
  
        par = ptr;	// Update parent pointer 
  
        // Moving on left subtree. 
        if (key < ptr->key) 
        { 
            if (ptr->l_thread == false) 
				child = ptr->left
                ptr = child
            else
                break; 
        } 
  
        // Moving on right subtree. 
        else
        { 
            if (ptr->r_thread == false) 
				child = ptr->right;
                ptr = child;
            else
                break; 
        } 
    } 

	lock(par)
	lock(ptr)
  
    // Create a new node 
    Node *tmp = new Node; 
    tmp->key = key; 
    tmp->l_thread = true; 
    tmp->r_thread = true; 
  
    if (par == NULL) 
    { 
        root = tmp; 
        tmp->left = NULL; 
        tmp->right = NULL; 
		unlock (par);
		return root;
    } 
    else if (key < (par->key)) 
    { 
		if (validate(par, ptr))
		{
			tmp->left = par->left; 
			tmp->right = par; 
			par->l_thread = false; 
			par->left = tmp; 
		}
    } 
    else
    { 
		if (validate(par, ptr))
		{
			tmp->left = par; 
			tmp->right = par->right; 
			par->r_thread = false; 
			par->right = tmp; 
		}
    } 

	unlock(par)
	unlock(ptr)
  
    return root; 
} 

// Insert a Node in Binary Threaded Tree 
struct Node *BSTSearch(struct Node *root, int key) 
{ 
    Node *ptr;
    Node *par;
    Node *child; 

    // Searching for a Node with given value 
	ptr = root

    while (ptr != NULL) 
    { 
        // If key already exists, return 
        if (key == (ptr->key)) 
        { 
			return ptr && !ptr. 
        } 

		if (!ptr->marked)
			return ptr;
		else 
			return NULL;
  
        // Moving on left subtree. 
        if (key < ptr->key) 
        { 
            if (ptr->l_thread == false) 
				child = ptr->left
                ptr = child
            else
                break; 
        } 
  
        // Moving on right subtree. 
        else
        { 
            if (ptr->r_thread == false) 
				child = ptr->right;
                ptr = child;
            else
                break; 
        } 
    } 
} 

// Delete a Node in Binary Threaded Tree 
struct Node *BSTDelete(struct Node *root, int key) 
{ 
    Node *ptr; // Current pointer
    Node *par; // Parent pointer
    Node *child; // Child pointer
	int found = 0;	// Key is found 

    // Searching for a Node with given value 
	lock(root);
	ptr = root

	// Search the key
    while (ptr != NULL) 
    { 
		if (dkey == ptr->key)
		{
			found = 1
			break;
		}

		par = ptr

		if (dkey < ptr->key)
		{
			if (ptr->l_thread == false)
			{
				child = ptr->left;
				lock(child);
				ptr = child;
			}
			else
				break;
		}
		else {
			if (ptr->r_thread == false)
			{
				child = ptr->right;
				lock(child);
				ptr = child;
			}
			else
				break;
		}

		unlock(par);
	} 

	if (found == 0)
		printf("The key does not found");

	else if (ptr->l_thread == false && ptr->r_thread == false)
	{
		struct Node* par_succ = ptr;
		struct Node* succ = ptr->right;

		lock(succ);

		while (succ->left != NULL)
		{
			parsucc = succ;
			succ = succ->left;
			lock(suck)
		}

		ptr->key = succ->key
	}

	else if (ptr->l_thread == false)
	{
		if (ptr->l_thread == false)
		{
			child = ptr->left
			lock(child)
		}
		else {
			child = ptr->right
			lock(child)
		}

		if (par == NULL)
			root = child;
		else if (ptr == par->left)
		{
			par->left = child;
		}
		else {
			par->right = child
		}

		Node *s = find_successor(ptr)
		lock(s)
		Node *p = find_predecessor(ptr)
		lock(p)

		if (validate(p, s))
		{
			// if ptr has left subtree
			if (ptr->l_thread == false)
				p->right = s

			// if ptr has right subtree
			else {
				if (ptr->r_thread == false)
					s->left = p
			}
		}

		free(ptr)
		unlock(par)
		unlock(child)
		unlock(p)
		unlock(s)

		return root;
	}

	else if (ptr->r_thread == false)
	{
		if (ptr->l_thread == false)
		{
			child = ptr->left
			lock(child)
		}
		else {
			child = ptr->right
			lock(child)
		}

		if (par == NULL)
			root = child;
		else if (ptr == par->left)
		{
			par->left = child;
		}
		else {
			par->right = child
		}

		Node *s = find_successor(ptr)
		lock(s)
		Node *p = find_predecessor(ptr)
		lock(p)
		
		if (validate(p, s))
		{
			// if ptr has left subtree
			if (ptr->l_thread == false)
				p->right = s

			// if ptr has right subtree
			else {
				if (ptr->r_thread == false)
					s->left = p
			}
		}

		free(ptr)
		unlock(par)
		unlock(child)
		unlock(p)
		unlock(s)

		return root;

	}
	else
	{
		// Delete root 
		struct Node* par_successor = ptr;
		struct Node* successor = ptr->right;
		lock(successor)

		while(successor->left != NULL)
		{
			par_successor = successor
			successor = successor->left

		}

		lock(par_successor)
		lock (successor)

		ptr->key = successor->key

		if (successor->l_thread == true && successor->r_thread == true)
		{
			struct Node* child;

			if (successor->l_thread == false)
			{
				child = successor->left
				lock(child)
			}
			else {
				child = successor->right
				lock(child)
			}

			if (par_successor == NULL)
				root = child;
			else if (successor == successor->left)
			{
				par_successor->left = child;
			}
			else {
				par_successor->right = child
			}

			Node *s = find_successor(successor)
			lock(s)
			Node *p = find_predecessor(successor)
			lock(p)

			if(validate(p, s))
			{
				// if successor has left subtree
				if (successor->l_thread == false)
					p->right = s

				// if successor has right subtree
				else {
					if (successor->r_thread == false)
						s->left = p
				}
			}

			free(ptr)
			unlock(par)
			unlock(successor)
			unlock(par_successor)
			unlock(child)
			unlock(p)
			unlock(s)

			return root;
		}
		// If the node to be deleted is left or its parent node
		else 
		{
			stuct Node* child
			if (successor->l_thread == false)
			{
				child = successor->left
				lock(child)
			}
			else {
				child = successor->right
				lock(child)
			}

			if (par_successor == NULL)
				root = child;
			else if (successor == par_successor->left)
			{
				par_successor->left = child;
			}
			else {
				par_successor->right = child
			}

			Node *s = find_successor(successor)
			lock(s)
			Node *p = find_predecessor(successor)
			lock(p)

			if(validate(p, s))
			{
				// if ptr has left subtree
				if (successor->l_thread == false)
					p->right = s

				// if ptr has right subtree
				else {
					if (successor->r_thread == false)
						s->left = p
				}
			}

			free(ptr)
			unlock(par)
			unlock(child)
			unlock(p)
			unlock(s)
			unlock(successor)
			unlock(par_successor)

			return root;
				
			}
	}

\end{lstlisting}

\paragraph{2.}
We implement a lazy synchronization locking for the lock-based
implementation of the concurrent threaded binary search tree. In our
implementation we lock the parent node, the current node and the child
node of each visit node. Before we add a new node we check the marked
bit if is enabled.

\paragraph{3.}
Let us have the scenario that I have a null tree, then add a new key
and then I delete this key. As a result we will have a null tree
again.
Liniarization points in Insertion: line 69 or line 88 where the new
element is already inserted in the tree.
Liniarization points in Deletion: after free operation when the nodes
are going to be unlocked.

\paragraph{4.}
The algorithm is correct because for each node that we traverse we
lock the parent node and the child node. By keeping these three nodes
locked during insertion and deletion operation we ensure that we are
going to perform correctly each one operation.

\paragraph{5.}
The algorithm is deadlock free because we lock only the current node
and its parent and child. So, all the threads can lock a node. Also by
locking individual nodes our threads can do progress in different
parts of the tree.

\paragraph{6.}
In cache-coherent NUMA architecture our algorithm on every lock/unlock
operation will produce invalidation to the other threads caches that
are waiting to get the lock for a node. Trying to use TTAS locks the
cache misses is equall to the numbers of set lock and unlock
operations we implement. In cache-less NUMA machines we will have
remote accesses for every check and update of the lock/unlock
operation. 
