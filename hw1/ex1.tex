\section*{Exercise A}

\paragraph{1a.}
In the scenario that we have two processes in the critical section (CS), 
the condition in line 4 must be true.

For example: state[1] != group and state[2] != group

But the group changes when a process gets out from the critical
section.

\paragraph{1b.}
Yes, this scenario is possible as we show:
We have two processes: P1 and P2. The P1 gets in the CS while the P2
is waiting in line 7. At this point, P1 gets out from the CS and it
changes the value of the group variable. Then, changes the state[i] =
3 and is executed until line 4. Then, it gets again to the CS.
However, P2 is still in waiting in line 7 because is much slower
process than P1.

\paragraph{1c.}
If we remove lines 1,5 then we do not guarantee that a process has
been enter to a group first and then will start to be executed.

\paragraph{1d.}
Line 4 is used to have fast access in the CS. If we remove line 4,
then we will have performance penalty because all the processes will
wait in line 7. Also, all the process will have to iterate all the
process in the loop.

\paragraph{2.}

