\section*{Exercise A}

\paragraph{1a.}
This algorithm presents the fair synchronization algorithm which uses
some key ideas from Black-White Bakery algorithm. $P_i$ to get in its
entry section of the critical section (CS) reads the value of the
group and determines which of the two groups (0 or 1) it should
belong. When the $P_i$ chooses a group, it waits until its group has
priority over the other group and then it enters to the CS. 
The order of which processes can enter to the CS is defined as
follows:%
(1) if two processes belong to two different groups, the process whose
group as recorded in its state is different from the value of group is
enabled and can enter to the CS and the other process has to wait, and
%
(2) if all the active processes belong to the same group then they can
all enter to the CS.
When a process exits from the CS set the group bit to a value which is
different from the value of its state. Thus, exit process ($P_i$)
gives priority to waiting processes which belong to the same group
that $P_i$ process belongs to.

\bigskip
Our algorithm needs to ensure for correct synchronization that two
waiting process $P_i$ and $P_j$, if $state[i] \neq group$ and
$state[j] = group$ then $P_i$ must enter the CS and completes its exit
section before $P_j$ enters its CS. 

\bigskip
A process $P_i$ is enabled to
enter its CS only when one of the following conditions are met: %
(1) the value of $state_i \neq group$, in such case, $P_i$ will break
out the for loop in line 4; or %
(2) for all processes $state[j] \neq 1 - state[i]$ where no process
belong to a different group than the group that $P_i$ belongs to. In
that case, $P_i$ will execute the loop $n$ times and will exit.
If none of these two conditions are true, $P_i$ will have to wait in
line 7, until the value of the group changes or the process $P_j$ of
that belongs to the other group changes its value.
The process $P_j$ can become enabled until the value of the group
change or all the processes of the group of $P_i$ complete their CS.

\paragraph{1b.}
Assume a beginning process $P_i$ overtakes a waiting process $P_j$ in
entering its critical section.  This can happen only if both $P_i$ and
$P_j$ belong to the same group at the time when $P_i$ has completed
executing line 2.  During the exit section of the CS $P_i$ will set
the value of the group to $1 - state[i]$.  Thereafter, by the
value of the group will not change until $P_j$ completes its exit
section.  If $P_i$ tries to enter to the CS again while $P_j$ has not
completed its exit section yet, then after passing through line 1
$P_i$ will belong to a different group than $P_j$ and the value of the
group bit will be the same as the value of $state[i]$. Thus $P_i$ will
not become enabled again until $P_j$ completes its exit section and
changes the value of its state $P_j$. Thus, the algorithm satisfies
fairness.

\paragraph{1c.}
If we omit line 1 and 5 will result in an incorrect algorithm. Line 1
and 5 ensures that a process has started executing its doorway code.
If we remove these two lines then a none beginning process $P_i$ may
enter in CS ahead of another waiting process $P_j$ twice: the first
time if $P_i$ is running on the other group and the second time if
$P_i$ pass $P_q$ which is waiting on the same group and enters in CS
first.

\paragraph{1d.}
Line 4 is used to ensure priority between the groups.

\paragraph{2.}
Solution for the first part of the exercise:

\begin{lstlisting}[style=mycode]

shared variables
int array[c], next_ticket, now_serving, desks, current_ticket, counter;

void hospital()
{
	int my_ticket	
	my_ticket = fetch_and_incr(next_ticket)

	while (now_serving != my_ticket);

	while (counter == 0);

	fetch_and_decr(counter)

	while (desks == 0);

	fetch_and_decr(desks)
	
	// Fill documents 

	fetch_and_incr(desks)

	int i = 0;

	while (true)
	{
		if (array[i].lock != 1)
		{
			lock(array[i])
			break;
		}
		i++;

		i (i >= c)
		{
			i = 0;
		}
	}

// CS

	unlock(array[i])

	fetch_and_incr(counter)
}

\end{lstlisting}

\paragraph{2.}
Solution for the second part of the exercise (graduate):

\begin{lstlisting}[style=mycode]
entry_section()
{

shared variables
int array[c], next_ticket, now_serving, desks, current_ticket, counter;

void hospital()
{
	int my_ticket	
	my_ticket = fetch_and_incr(next_ticket)

	while (now_serving != my_ticket);

	while (counter == 0);

	fetch_and_decr(counter)

	while (desks == 0);

	fetch_and_decr(desks)
	
	// Fill documents 

	fetch_and_incr(desks)

	int i = 0;

	while (true)
	{
		if (array[i].lock != 1)
		{
			lock(array[i])
			break;
		}
		i++;

		i (i >= c)
		{
			i = 0;
		}
	}

	while (worker[my_ticket % w].lock == 1);
	
	fetch_and_incr(worker[my_ticket % w].lock);

	fetch_and_incr(samples)

	unlock(array[i])

	fetch_and_incr(counter)
}

\end{lstlisting}
