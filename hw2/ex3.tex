\section*{Exercise 3A(i)}

\begin{lstlisting}[style=mycode]
Initialy each process has:
v = id
status = active
phase = 0;
received = 0
rcv_val[2] = {0, 0} // rcv_val[1]: Value of the first left node 
					// rcv_val[0]: Value of the second left node
leader = false;

upon receiving no message:
	if (status == active )
		send <M, v, phase, 2> to right

upon receiving <M, value, phaseId, counter> from left
	if (status == active) {
		if (counter > 0) {
			received++
			counter--
			rcv_val[counter] = value
			send <M, value, phaseId, counter> to right
		}

		if (received == 2 && state == active) {
			if (rcv_val[1] == v) {
				received = 0
				leader = true
				terminate
			}

			if (rcv_val[1] > rcv_val[0] && rcv_val[1] > v) {
				v  = rcv_val[1];
				phase++;
				received = 0
				send <M, v, phase, 2> to right
			}
			else {
				state = relay 
			}
		}
	}
	else
		send (M, value, phaseId, counter) to right
\end{lstlisting}

\section*{Exercise 3A(ii)}
\paragraph{Message complexity:} $O(n^2)$, in the worst case because we
send message to all nodes in the right. Although innactive nodes
forward messages to the right.

\section*{Exercise 3A(iii)}
\paragraph{Time complexity:} $O(n)$, the time needed by the last node
in the ring to get the message. So, is the time that needs a message
to traverse the entire ring. 

\section*{Exercise 3B(i)}

\begin{lstlisting}[style=mycode]

Initialy each process has:
v = id
status = active
phase = 0;
received = 0
rcv_val[2] = {0, 0} // rcv_val[1]: Value of the right node 
					// rcv_val[0]: Value of the left node
leader = false;

upon receiving no message
	if (status == active)
		rcv_val[0] = id of the left node
		rcv_val[1] = id of the right node

		if (rcv_val[0] > rcv_val[1] && rcv_val[0] > v)
			v = rcv_val[0]
			phase++
			rcv_val[0] = -1
			rcv_val[1] = -1
			send <M, 0, -1> to left and right nodes
		else
			state = relay

upon receiving <M, counter, val> from right
	if (state == relay)
		if (val == -1)
			counter++
			send <M, counter, val> to left
		else
			counter--
			send <M, counter, val> to right

	else
		if (count > 0)
			count--
			send <M, counter, v> to right
		else 
			if (rcv_val[0] != -1 &&  rcv_val[1] != -1) {
				if (rcv_val[0] == v)
					terminate as the leader

				if (rcv_val[0] > rcv_val[1] && rcv_val[0] > v)
					v = rcv_val[0]
					phase++
					rcv_val[0] = -1
					rcv_val[1] = -1
					send <M, 0, -1> to left and right nodes
				else
					state = relay
			}
			else
				rcv_val[1] = val

upon receiving <M, counter, val> from left
	if (state == relay)
		if (val == -1)
			counter++
			send <M, counter, val> to right
		else
			counter--
			send <M, counter, val> to left

	else
		if (count > 0)
			count--
			send <M, counter, v> to left
		else 
			if (rcv_val[0] != -1 &&  rcv_val[1] != -1) {
				if (rcv_val[0] == v)
					terminate as the leader

				if (rcv_val[0] > rcv_val[1] && rcv_val[0] > v)
					v = rcv_val[0]
					phase++
					rcv_val[0] = -1
					rcv_val[1] = -1
					send <M, 0, -1> to left and right nodes
				else
					state = relay
			}
			else
				rcv_val[0] = val

\end{lstlisting}

\section*{Exercise 3B(ii)}
\paragraph{Message complexity:} $O(n^2)$, in the worst case because we
send message to all nodes in the left and right. Although innactive nodes
forward messages to the right.

\section*{Exercise 3B(iii)}
\paragraph{Time complexity:} $O(n)$, the time needed by the last node
in the ring to get the message. So, is the time that needs a message
to traverse the entire ring. 
