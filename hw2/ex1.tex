\section*{Exercise 1(i)}

\begin{lstlisting}[style=mycode]
Every process (p) has the following variables

var depth = 0;
var replies = 0;
var num_leafs = 0;

// First construct spanning tree. The following four methods are used
to construct the rooted spanning tree.
upon receiving no message:
	if pi == pr && parent == null then
		send <M> to all neighbors
		parent = pi

upon receiving <M> from neighbor pj:
	if parent == NULL then
		parent = pj
		send <parent> to pj
		if neighbors.size() > 1 then
			depth = 1;
			send <M> to all neighbors except pj
	else
		send <already> to pj

upon receiving <parent> from neighbor pj:
	add pj to children
	if children U other contains all neighbors except parent then
		terminate

upon receiving <already> from neighbor pj:
	add pj to other
	if children U other contains all neighbors except parent then
		terminate

// Calculate the number of leafs in each subtree and then broadcast
// the value to the leader node. The leader node will decide if
// the leafs are in the same level
upon receive no message:
	if pi == pr && parent == null then
		send <M> to all childrens

upon receiving <M> from parent pj
	if has_no_children 
		send <Leaf, 1, depth> to parent pj
	else 
		send <M> to all childrens

upon receiving <Leaf, count, d> from children pj
	replies ++;
	num_leafs += count

	depth -= d

	if replies == childrens
		if parent == null
			// If the depth == 1 that means that all leafs are in the
			// same level. So broadcast this information to all
			// processes in the spanning tree.
			if (depth == 1)
				send<LEVEL, true> to all children
			else 
				send<LEVEL, false> to all children
		else 
			if (depth == 1)
				send <Leaf, num_leafs, 0> to parent
			else 
				send <Leaf, num_leafs, depth> to parent

upon receiving <LEVEL, is_same_level> from parent pj
	send<LEVEL, is_same_level> to all children
	
\end{lstlisting}

\section*{Exercise 1(ii): Communication complexity}

\paragraph{Creation of Spanning Tree:} O(m*n)

\paragraph{Count Number of Leafs:} $O(n - 1 + n + n - 1)$, where
$\textbf{(n - 1)}$ are the messages that are sent from the root to all
the processes to count the number of leafs and calculate the depth,
%
$\textbf{(n)}$ is the total messages send from the leafs until the
root calculating the number of leafs in each subtree and check if the
leafs are on the same level, and
%
$\textbf{(n - 1)}$ are the total number of messages that are sent from
the root to the leafs of the spanning tree to inform all the processes
if the leafs are on the same level.

\medskip
So, $\textbf{Communication Complexity = O(3n - 2) = O(n)} $

\section*{Exercise 1(iii): Time complexity}
\paragraph{Creation of Spanning Tree:} $O(d)$, where $\textbf{d}$ is
the diameter of the graph. The length of the longest of the shortest
path between any two vertices.
\paragraph{Count Leafs and depth:} $O(h)$, where $\textbf{h}$ is the
height of the tree (the length of the longest path from the root to
leaf)

\medskip
So, $\textbf{Time Complexity = O(d + h)} $
